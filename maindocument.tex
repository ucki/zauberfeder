\documentclass[a4paper, 12pt, oneside]{article}
\usepackage[utf8x]{inputenc}
\usepackage[T1]{fontenc}
\usepackage{palatino}
\linespread{1.25}
\usepackage{microtype}
\usepackage[english]{babel}
%More citations
\usepackage{natbib}
%better urls mit \url{}
\usepackage{url}
\usepackage{lastpage}
%code
\usepackage{listings}

%more colors
\usepackage{soul}

%Force figure to be placed “HERE”
\usepackage{float} 


%folgende Zeilen sind für Kapitelüberschriften
\usepackage[rigidchapters]{titlesec}
\usepackage{blindtext}
\titleformat{\chapter}
{\normalfont\LARGE}
{\makebox[3pc][l]{\LARGE\thechapter\hfil\rule[-6pt]{0.5pt}{2pc}}}
{0pt}
{\LARGE}
\titlespacing*{\chapter}{0pt}{0pt}{82pt}

%Csv -> Latex
\usepackage{csvsimple}

%Für Graphiken 
\usepackage{tikz}
\usetikzlibrary{plotmarks}
\usetikzlibrary{positioning,shapes,shadows,arrows}
\usepackage{graphicx}

%Paket gibt einige Optionen mehr bei Tabellen (wird eher nicht verwendet)
\usepackage{array}
%Seitenränder definieren
\usepackage[right=4.0cm,left=3.3cm, bottom=3.9cm, top=4.1cm, footskip=2.1cm, headsep=2.0cm]{geometry}
%\usepackage{stdpage}
%test wegen anzahl zeilen pro seite
%Paket für Zeilenabstände
\usepackage{setspace}
%\onehalfspacing
\usepackage{multirow}
%Paket gibt mehr Kontrolle über die Captions (Bildunterschriften) bei Abbildungen
\usepackage[labelfont=bf,format=hang,font=footnotesize,justification=raggedright,singlelinecheck=false]{caption}

%Helvetia (Arial) Verwenden WICHTIG: Beide folgenden Zeilen kopieren!
%\usepackage[scaled]{helvet} %
%\renewcommand*\familydefault{\sfdefault} %%

% Abschalten des Einrückens bei neuen Absätzen (manuell, nach Tabellen, Abbildungen, etc.)
\setlength{\parindent}{0pt}
\hyphenation{}

\newcommand{\sectionbreak}{\clearpage}

\usepackage{color}

\definecolor{color0}{rgb}{0,0,0}% black
\definecolor{color1}{rgb}{0.22,0.45,0.70}% light blue
\definecolor{color2}{rgb}{0.45,0.45,0.45}% dark grey
\definecolor{mygreen}{rgb}{0,0.6,0}
\definecolor{mygray}{rgb}{0.5,0.5,0.5}
\definecolor{myblue}{rgb}{0.0, 0.53, 0.74}
\definecolor{codebackground}{rgb}{0.8, 0.8, 0.8}
\definecolor{codechanged}{rgb}{0.8, 0.0, 0.0}

\lstset{ %
  backgroundcolor=\color{codebackground},   % choose the background color; you must add \usepackage{color} or \usepackage{xcolor}
  basicstyle=\footnotesize,        % the size of the fonts that are used for the code
  breakatwhitespace=false,         % sets if automatic breaks should only happen at whitespace
  breaklines=true,                 % sets automatic line breaking
  captionpos=b,                    % sets the caption-position to bottom
  commentstyle=\color{mygreen},    % comment style
  deletekeywords={...},            % if you want to delete keywords from the given language
  escapeinside={\%*}{*)},          % if you want to add LaTeX within your code
  extendedchars=true,              % lets you use non-ASCII characters; for 8-bits encodings only, does not work with UTF-8
% frame=single,	                   % adds a frame around the code
  keepspaces=true,                 % keeps spaces in text, useful for keeping indentation of code (possibly needs columns=flexible)
% keywordstyle=\color{blue},       % keyword style
% language=Octave,                 % the language of the code
% otherkeywords={*,...},           % if you want to add more keywords to the set
  numbers=left,                    % where to put the line-numbers; possible values are (none, left, right)
  numbersep=5pt,                   % how far the line-numbers are from the code
  numberstyle=\tiny\color{mygray}, % the style that is used for the line-numbers
  rulecolor=\color{black},         % if not set, the frame-color may be changed on line-breaks within not-black text (e.g. comments (green here))
  showspaces=false,                % show spaces everywhere adding particular underscores; it overrides 'showstringspaces'
  showstringspaces=false,          % underline spaces within strings only
  showtabs=false,                  % show tabs within strings adding particular underscores
  stepnumber=2,                    % the step between two line-numbers. If it's 1, each line will be numbered
%  stringstyle=\color{mymauve},     % string literal style
  tabsize=2,	                   % sets default tabsize to 2 spaces
% title=\lstname,                   % show the filename of files included with \lstinputlisting; also try caption instead of title
  moredelim=**[is][\color{codechanged}]{**@}{@**}, %Rot für Änderungen
  moredelim=**[is][\color{myblue}]{***@}{@***}, %einfaches blau
  moredelim=**[is][\color{mygreen}]{*@}{@*},%einfaches Grün
}

%To toggle the priv escalation part
\usepackage{environ}
\usepackage{etoolbox}
%Aktives Inhaltsverzeichnis und links
\usepackage{hyperref}
\hypersetup{
    colorlinks,
    citecolor=black,
    filecolor=black,
    linkcolor=black,
    urlcolor=black
}


%---------------------------------------------------------------------------------
%--------------------------Set the variables for the Mainpage here----------------
%---------------------------------------------------------------------------------

\newcommand{\email}{john@doe.com}
\newcommand{\osid}{OS-11111}
\newcommand{\name}{John Doe}


\title{{\textbf{\Huge Offensive Security}}\\
	Penetration Test Report for\\Internal Lab and Exam}
\author{\vspace{3cm}\\{\LARGE \name}\\[1em]\email\\[1em]OSID: \osid}
\date{\vspace{7cm}\today}

\usepackage{fancyhdr}
\pagestyle{fancy}
\fancypagestyle{plain}{}
\fancyhf{}
\fancyfoot{} % clear all footer fields

\fancyhead[R]{\small{\rightmark}}
\fancyhead[L]{\small{Offensive Security - Penetration Test Report}}

\renewcommand{\footrulewidth}{0.1pt} % Create a rule above the page number
\fancyfoot[R]{\textcolor{color1}\thepage  \textcolor{color2}{/\pageref{LastPage}}}



\begin{document}
\maketitle
\thispagestyle{empty}
\tableofcontents
\thispagestyle{empty}
\pagebreak


%--------------------------------Commands in this document
%\url{http://blafasel.com} makes nice clickable links in the text
%\FloatBarrier stops floating images at this point.

%In the code section :
% *@Green@*
% **@Red Text@**
% ***@Blue Text@***

%--------------------------Document might need some Runs to get all Indexes etc right .. usual Latx stuff :)

%--------------------------Declaring the variables for the hosts

\newcommand{\hostname}{}
\newcommand{\ip}{}
\newcommand{\tcpports}{} 
\newcommand{\udpports}{} 
\newcommand{\os}{}
\newcommand{\vuln}{}
\newcommand{\product}{}
\newcommand{\vulnx}{}
\newcommand{\productx}{}

 
%-------------------------Here are the hosts, just add in the same sheme after you coppied the example folder ---------------------------------------------------------

% !TeX spellcheck = en_US

%---------------------------------------------------------------------------------
%--------------------------Set the variables for every client---------------------
%---------------------------------------------------------------------------------
\renewcommand{\hostname}{Example}
\renewcommand{\os}{Linux OS Soft XP}
\renewcommand{\ip}{42.42.42.42}
\renewcommand{\tcpports}{1,2,3}
\renewcommand{\udpports}{23,42}
\renewcommand{\vuln}{CVE-123-42 \glqq Stupid Idiot User\grqq}
\renewcommand{\product}{Human}
%The x-Variables are only used, when root is defined (== root shell)
\renewcommand{\vulnx}{CVE-123-43 \glqq Very Stupid Idiot User Again\grqq} 
\renewcommand{\productx}{Human}
%%%Did you get root? Comment out if you only got low priv access
\def\gotroot{}   %%% Define root if you got root shell
%\undef\gotroot % Else undefine


%----------------------------------------------------------------------------------
%-------------------------------Auto generated content-----------------------------
%----------------------------------------------------------------------------------

\section{\hostname}
\subsection{Service Enumeration}

\begin{table}[h]
	\begin{tabular}{|c|c|}
		\hline
		\multicolumn{2}{|c|}{\textbf{\hostname}}\\\hline\hline
		Type         & Open ports   \\\hline
		TCP          & \tcpports{}  \\\hline
		UDP          & \udpports{}  \\\hline\hline
		\textbf{\os} & \textbf{\ip} \\\hline
	\end{tabular}
	\caption{Service enumeration \hostname}
\end{table}

\subsection{Remote Access Exploitation}

\paragraph{Vulnerability Exploited:}
\vuln

%----------------------------------------------------------------------------------
%-------------------------------Start writing here---------------------------------
%----------------------------------------------------------------------------------
 
\paragraph{Vulnerability Explanation:}
% You can answer the following questions:
% What is the problem?
% What can an attacker do with this vulnerability?
% How did you notice this bug?
% Did you need to change an exploit in order to run it?
% What was the result of the execution?


\paragraph{Vulnerability Fix:}
The publishers of \product{} have issued a patch to fix this known issue.

\paragraph{Severity:}
\textbf{\textcolor{red}{Critical}}

\paragraph{Proof of Concept:} 
Modifications to the existing exploit was needed and is highlighted in red.
\begin{lstlisting}[caption={Exploitation of \hostname}]
SELECT * FROM login WHERE id = **@1 or 1=1@** AND user LIKE "%root%"

if (x = y):
    space indent
	tab indent

In the code section :
*@Green Text@*
**@Red Text@**
***@Blue Text@***
\end{lstlisting}

\begin{figure}[H]
	\centering
	\includegraphics [width=1.0\textwidth]{./hosts/\hostname/1-remote-exploit.png}
	%scale 0.5 bedeutet 50% der originalgröße
	%angle=90 Grafik um 90° drehen
	\caption{Exploitation of \hostname}
\end{figure}

\paragraph{Proof of remote access:} %Print proof for local exploit here
The remote access can be proven with the following command:
\begin{lstlisting}[caption={Post exploitation of \hostname{} with low privileges}]
hostname && id && ifconfig && cat local.txt
\end{lstlisting}
\begin{figure}[H]
	\centering
	\includegraphics [width=\textwidth]{./hosts/\hostname/2-local.png}
	\caption{Proof of remote access to \hostname}
\end{figure}

%----------------------------------------------------------------------------------
%------------------------------Conditional Privilege Escalation Block--------------
%----------------------------------------------------------------------------------
\ifdefined\gotroot

%----------------------------------------------------------------------------------
%------------------------------Part for Priv escalation----------------------------
%----------------------------------------------------------------------------------
\subsection{Privilege Escalation}

\paragraph{Vulnerability Exploited:}
\vulnx

\paragraph{Vulnerability Explanation:}
% You can answer the following questions:
% What is the problem?
% What can an attacker do with this vulnerability?
% How did you notice this bug?
% Did you need to change an exploit in order to run it?
% What was the result of the execution?


\paragraph{Vulnerability Fix:}
The publishers of \product{} have issued a patch to fix this known issue.

\paragraph{Severity:}
\textbf{\textcolor{red}{Critical}}

\paragraph{Proof of Concept:} 
Modifications to the existing exploit was needed and is highlighted in red.
\begin{lstlisting}[caption={Exploitation of \hostname}]
SELECT * FROM login WHERE id = **@1 or 1=1@** AND user LIKE "%root%"
In the code section :
*@Green Text@*
**@Red Text@**
***@Blue Text@***
\end{lstlisting}

\begin{figure}[H]
	\centering
	\includegraphics [width=1.0\textwidth]{./hosts/\hostname/3-privesc-exploit.png}
	\caption{Privilege escalation exploit of \hostname}
\end{figure}

\paragraph{Proof of successful privilege escalation:}
The successful privilege escalation can be proven with the following command:
\begin{lstlisting}[caption={Post exploitation of \hostname}]
hostname && id && ifconfig && cat proof.txt
\end{lstlisting}

\begin{figure}[H]
	\centering
	\includegraphics [width=\textwidth]{./hosts/\hostname/4-proof.png}
	\caption{Proof of successful privilege escalation on \hostname}
\end{figure}
\fi % End of if-block
%----------------------------------------------------------------------------------
%--------------------------------------End of Priv Esc Block-----------------------
%----------------------------------------------------------------------------------



%---------------------------------------------------------------------------------
%--------------------------Set the variables for every client---------------------
%---------------------------------------------------------------------------------
\renewcommand{\hostname}{DeepThought}
\renewcommand{\os}{Earth}
\renewcommand{\ip}{42.42.42.23}
\renewcommand{\tcpports}{1,2,3}
\renewcommand{\udpports}{23,42}
\renewcommand{\vuln}{Vogons}
%%%Did you get root? Comment out if you only got low priv access
\def\root{}   %%% Define root if you got root shell

%----------------------------------------------------------------------------------
%-------------------------------Auto generated content-----------------------------
%----------------------------------------------------------------------------------

\section{\hostname}
\subsection{Service Enumeration}

\begin{table}[h]
	\begin{tabular}{|c|c|}
		\hline
		\multicolumn{2}{|c|}{\textbf{\hostname}}\\\hline\hline
		Type         & Open ports   \\\hline
		TCP          & \tcpports{}  \\\hline
		UDP          & \udpports{}  \\\hline\hline
		\textbf{\os} & \textbf{\ip} \\\hline
	\end{tabular}
	\caption{Service enumeration \hostname}
\end{table}

\subsection{Remote Access Exploitation}

\paragraph{Vulnerability Exploited:}
\vuln

%----------------------------------------------------------------------------------
%-------------------------------Start writing here---------------------------------
%----------------------------------------------------------------------------------

\paragraph{Vulnerability Explanation:}
BLA BLA WRITE SOMETHING HERE 



\paragraph{Severity:}
\textbf{\textcolor{red}{Critical}}

\paragraph{Proof of Concept:}
Some text here
\begin{lstlisting}[caption={Exploitation of \hostname}]
*@Kali prep:@*

*@Modifications in the exploit@*
PANIC **@PANIC@** PANIC
*@Running the exploit@*

*@Escaping the low priv shell:@*
\end{lstlisting}


\begin{figure}[H]
\centering
\includegraphics [width=\textwidth]{./hosts/\hostname/exploitexecution.png}
%scale 0.5 bedeutet 50% der originalgröße
%angle=90 Grafik um 90° drehen
\caption[Exploitation of \hostname]{Exploitation of \hostname} \label{\hostname-1}
\end{figure}


\ifdefined\root
   

%Conditional Privilege Escalation Block .. --------------------------------------------------------------------------------------------------------------------


%Part for Priv escalation -------------------------------------------------------------------------------------------
\subsubsection{Privilege Escalation}




\begin{figure}[H]
\centering
\includegraphics [width=\textwidth]{./hosts/\hostname/local.png}
%scale 0.5 bedeutet 50% der originalgröße
%angle=90 Grafik um 90° drehen
\caption[Local shell of \hostname]{Local shell of \hostname} \label{\hostname-2}
\end{figure}


\begin{figure}[H]
\centering
\includegraphics [width=\textwidth]{./hosts/\hostname/privescexploit.png}
%scale 0.5 bedeutet 50% der originalgröße
%angle=90 Grafik um 90° drehen
\caption[Priv escalation exploit of \hostname]{Priv escalation exploit of \hostname} \label{\hostname-3}
\end{figure}

\fi
%--------------------------------------End of Priv Esc Block------------------------

\subsubsection{Proof and Post escalation}
\begin{lstlisting}[caption={Post exploitation of \hostname},label=\hostname-post]
*@Post exploitation commands run:@*

\end{lstlisting}

\begin{figure}[H]
\centering
\includegraphics [width=\textwidth]{./hosts/\hostname/proof.png}
%scale 0.5 bedeutet 50% der originalgröße
%angle=90 Grafik um 90° drehen
\caption[Proof of \hostname]{Proof of \hostname} \label{\hostname-4}
\end{figure}


%---------------------End of host file

\end{document}
